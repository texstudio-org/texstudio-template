\documentclass[10pt]{article}
\usepackage[utf8]{vietnam}
\usepackage[margin=.5in]{geometry}
\usepackage{amsmath, amssymb, amsthm, gensymb, multicol, graphicx, ragged2e, nopageno, wrapfig, charter, tkz-euclide, rotating}
\usepackage[colorlinks=true, urlcolor=blue, linkcolor=red]{hyperref}
\usepackage[framemethod=1]{mdframed}
\usetikzlibrary{math}
\newcommand{\fcb}[1]{\fcolorbox{red}{yellow}{\textbf{\textcolor{blue}{#1}}}} % Viền đỏ, nền vàng, chữ xanh đậm
\newcommand{\ctext}[1]{\huge{\textbf{#1}}} % Tiêu đề
% Các tập sau đây thường được dùng, viết rút gọn
\newcommand{\R}{\mathbb{R}} % Tập số thực
\newcommand{\Rp}{\mathbb{R}_+} % Tập số thực dương
\newcommand{\Z}{\mathbb{Z}} % Tập số nguyên
\newcommand{\Zp}{\mathbb{Z}_+} % Tập số nguyên dương
\newcommand{\N}{\mathbb{N}} % Tập số tự nhiên
\newcommand{\Q}{\mathbb{Q}} % Tập số hữu tỉ
\newcommand{\Qp}{\mathbb{Q}_+} % Tập số hữu tỉ dương
\newcommand{\PR}{\mathbb{P}} % Tập số nguyên tố
\newcommand{\opintvl}[2]{\left(#1; #2\right)} % Khoảng
\newcommand{\clointvl}[2]{\left[#1; #2\right]} % Đoạn
\newcommand{\lclointvl}[2]{\left[#1; #2\right)} % Khoảng nửa đóng (bên trái)
\newcommand{\rclointvl}[2]{\left(#1; #2\right]} % Khoảng nửa đóng (bên phải)
\newcommand{\power}[2]{\mathcal{P}\left(#1, #2\right)} % Phương tích của điểm đến đường tròn
\newcommand{\fphi}[1]{\phi\left(#1\right)} % Hàm phi - Euler
\newcommand{\ftau}[1]{\tau\left(#1\right)} % Hàm đếm ước
\newcommand{\fsigma}[2]{\sigma_{#1}\left(#2\right)} % Hàm tổng lũy thừa ước
\newcommand{\fpi}[1]{\pi\left(#1\right)} % Hàm đếm số nguyên tố
%%%%%%%%%%%%%%%% Tiêu đề, tác giả và ngày tháng
\title{\ctext{Tiêu đề ở đây}} % Tiêu đề
\author{\ctext{Hoàng Văn Phú}} % Tác giả
\date{\textbf{Ngày 8 tháng 11 năm 2024}}
%%%%%%%%%%%%%%%% Bắt đầu văn bản
\begin{document}
	\pagestyle{empty} % Không đánh dấu trang, có thể bỏ đi cả nopageno ở \usepackage và dòng này
	\RaggedRight
	\Large
	\fontfamily{bch}\selectfont
	\boldmath
	\theoremstyle{definition}
	\newmdtheoremenv[innerleftmargin = 4mm,	innerrightmargin = 4mm,	innertopmargin = 2mm, innerbottommargin = 4mm, linecolor=red, backgroundcolor=yellow, leftline=false, rightline=false]{theorem1dl}{Định lý}
	\newmdtheoremenv[innerleftmargin = 4mm,	innerrightmargin = 4mm,	innertopmargin =2mm, innerbottommargin = 4mm, linecolor=red, backgroundcolor=yellow!50, leftline=false, rightline=false]{theorem2bd}{Bổ đề}
	\newmdtheoremenv[innerleftmargin = 4mm,	innerrightmargin = 4mm,	innertopmargin = 2mm, innerbottommargin = 4mm, linecolor=red, backgroundcolor=yellow!75, leftline=false, rightline=false, linewidth=2pt]{theorem3nx}{Nhận xét}
	\maketitle
		
		\maketitle
		
		\begin{abstract}
			Nội dung tóm tắt nên được ghi ở đây
		\end{abstract}
		
		\section{Phân mục ở đây}
		Đây là 1 văn bản thử nghiệm
	
\end{document}
